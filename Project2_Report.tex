%--------------------------------------------
%%%%%%%%%%%%%%%%%%%%%%%%%%%%%%%%%%%%%%%%%%%%%%%%%%%%
% THE PREAMBLE
%%%%%%%%%%%%%%%%%%%%%%%%%%%%%%%%%%%%%%%%%%%%%%%%%%%%
%--------------------------------------------

%%%%%%%%%%%%%%%%%%%%%%%%%%%%%%%%%%%%%%%%%%%%%%%%%%%%
% DOCUMENT CLASS (don't ever change this unless specified to do so).
%%%%%%%%%%%%%%%%%%%%%%%%%%%%%%%%%%%%%%%%%%%%%%%%%%%% 
\documentclass[11pt]{article}


%%%%%%%%%%%%%%%%%%%%%%%%%%%%%%%%%%%%%%%%%%%%%%%%%%%%
% USEFUL PACKAGES THAT YOU NEED. (Everything here needs to be here for the program to run.  Do not delete anything from the preamble, but you can add new things that you find on the internet.)
%%%%%%%%%%%%%%%%%%%%%%%%%%%%%%%%%%%%%%%%%%%%%%%%%%%%
\usepackage{fullpage}
\usepackage{amsfonts}
\usepackage{amsmath}
\usepackage{amsthm}
\usepackage{graphicx}
\usepackage{color}
\usepackage{amssymb}
\usepackage{empheq}
\usepackage{mathrsfs}
\usepackage{enumerate}
\usepackage{tikz}
\usepackage{pgflibraryarrows}
\usepackage{pgflibrarysnakes}
\usepackage{upgreek}
\usepackage{tipa}
\usepackage{gensymb}
\usepackage{multicol}
\usepackage{textcomp}
\usepackage{float}
\usepackage{ulem}
\usepackage{setspace}
\usepackage[margin=1in]{geometry}
\usepackage{amsmath}
\usepackage{graphicx}
\usepackage{wrapfig}
\usepackage{matlab-prettifier}
\usepackage{listings}
% Define a custom style for MATLAB code using matlab-prettifier
\lstdefinestyle{Matlab-editor}{
    style=Matlab-Pyglike, % Use a nice default style from the package
    basicstyle=\mlttfamily\small, % Use a small, typewriter font
    frame=single,                 % Add a frame around the code
    numbers=left,                 % Add line numbers on the left
    numberstyle=\tiny\color{gray} % Make line numbers small and gray
}
\usepackage{tikz}
\newcommand*\circled[1]{\tikz[baseline=(char.base)]{
    \node[shape=circle,draw,inner sep=1pt] (char) {#1};}}
% Tables: nicer rules and multi-row cells
\usepackage{booktabs}
\usepackage{multirow}
\usepackage{pifont}
\newcommand{\cmark}{\ding{51}} % check
\newcommand{\xmark}{\ding{55}} % cross
\setlength{\parindent}{0pt}

%%%%%%%%%%%%%%%%%%%%%%%%%%%%%%%%%%%%%%%%%%%%%%%%%%%%
% CREATES A SOLUTION ENVIRONMENT FOR HOMEWORK ASSIGNMENTS (You may not use this but leave it here for future assignments)
%%%%%%%%%%%%%%%%%%%%%%%%%%%%%%%%%%%%%%%%%%%%%%%%%%%%
\usepackage{versions}
\excludeversion{sol}
\includeversion{sol}
\newenvironment{solution}{
\sol\\{\sc{Solution:}}}{
$\hfill\blacksquare$\endsol}


%%%%%%%%%%%%%%%%%%%%%%%%%%%%%%%%%%%%%%%%%%%%%%%%%%%%
% CUSTOM DEFINITIONS FOR CREATING DERIVATIVE SYMBOLS (You will definitely use these.)
%%%%%%%%%%%%%%%%%%%%%%%%%%%%%%%%%%%%%%%%%%%%%%%%%%%%
\newcommand{\pd}[2]{\frac{\partial #1}{\partial #2}}
\newcommand{\pdd}[2]{\frac{\partial^2 #1}{\partial {#2}^2}}
\newcommand{\de}[2]{\frac{d #1}{d #2}}
\newcommand{\dde}[2]{\frac{d^2 #1}{d #2^2}}
\renewcommand{\figurename}{Plot}

%%%%%%%%%%%%%%%%%%%%%%%%%%%%%%%%%%%%%%%%%%%%%%%%%%%%
% CREATES THE TITLE PAGE (Note where to put the author, title, and date.)
%%%%%%%%%%%%%%%%%%%%%%%%%%%%%%%%%%%%%%%%%%%%%%%%%%%%
\title{\textsc{Dylan Huston \\ Isai Moreno}}
\author{\Large{ECE 3793 -- \textsc{Project 2}}}
\date{November 18, 2025}


%%%%%%%%%%%%%%%%%%%%%%%%%%%%%%%%%%%%%%%%%%%%%%%%%%%%
% CREATES A HEADER FOR EACH PAGE.  Please note where the left, center, and right titles are:
%%%%%%%%%%%%%%%%%%%%%%%%%%%%%%%%%%%%%%%%%%%%%%%%%%%%
\usepackage{fancyhdr}
\setlength{\headheight}{15.2pt}
\pagestyle{fancy}
\setlength\headsep{30pt}
\lhead{Dylan Huston | Isai Moreno}   					
\chead{\textsc{Project 2}}			    %  Title in the center.
\rhead{\today}							%  Date on the right header.

\begin{document}

\maketitle
\pagebreak

%--------------------------------------------
%%%%%%%%%%%%%%%%%%%%%%%%%%%%%%%%%%%%%%%%%%%%%%%%%%%%
% END OF THE PREAMBLE AND BEGINNING OF THE ACTUAL DOCUMENT
%%%%%%%%%%%%%%%%%%%%%%%%%%%%%%%%%%%%%%%%%%%%%%%%%%%%
%--------------------------------------------


%%%%%%%%%%%%%%%%%%%%%%%%%%%%%%%%%%%%%%%%%%%%%%%%%%%%
% SECTIONS, SUBSECTIONS, EQUATIONS, and OTHER THINGS:
%%%%%%%%%%%%%%%%%%%%%%%%%%%%%%%%%%%%%%%%%%%%%%%%%%%%

\section*{Introduction}

In this project, our team will be deriving and applying Fourier constants across a simple step function. During this process several MATLAB calculations will be performed, as well as deriving the Fourier constant integral to be used in recreating a signal. These signals and constants will then be plotted. The code that is used to derive the Fourier constants will then be used to derive the constants of different instruments playing the C3 note.  


\pagebreak

\section*{Part I: Derivation for the Transmitter}

Given a signal with a frequency of 500 Hz, \\ 
\begin{equation}
    x(t) = \text{cos}(2\pi \times 500t)
\end{equation}
and an arbitrary carrier frequency (\(F_c\)), to which the carrier signal is
\begin{equation}
    x_c(t) = \text{cos}(2\pi F_ct)
\end{equation}

\subsection*{Calculation 1: \(x_{TX}(t)\) as a sum of cosines}

\begin{equation}
x_{TX}(t) = x(t)x_c(t)
\end{equation}

Using the known signals, \(x(t)\) and \(x_c(t)\), we can substitute these functions into \(x_{TX}(t)\).

\[x_{TX}(t) = (\text{cos}(2\pi \times 500t))(\text{cos}(2\pi F_ct))\]

Using the trigonometric identity
\begin{equation} 
\text{cos}(u)\text{cos}(v) = \dfrac{1}{2}[\text{cos}(u+v)+\text{cos}(u-v)]
\end{equation}

Substituting in equation 1 for the cos(u) term, and equation 2 for the cos(v) term,

   \[\text{cos}(2\pi \times 500)\text{cos}(2\pi F_ct) = \dfrac{1}{2}[\text{cos}(2\pi \times 500t + 2\pi F_ct) + \text{cos}(2\pi \times 500t - 2\pi F_ct)]\]


Resulting in the equation

\begin{equation}   
x_{TX}(t) = \dfrac{1}{2}[\text{cos}(2\pi \times 500t + 2\pi F_ct) + \text{cos}(2\pi \times 500t - 2\pi F_ct)]
\end{equation}

Then distributing the \(1/2\) to each cosine

\[x_{TX}(t) = \dfrac{1}{2}\text{cos}(2\pi \times 500t + 2\pi F_ct) + \dfrac{1}{2}\text{cos}(2\pi \times 500t - 2\pi F_ct)\]

and factoring out \(2\pi\) from each cosine term

\[x_{TX}(t) = \dfrac{1}{2}\text{cos}(2\pi(500t + F_ct)) + \dfrac{1}{2}\text{cos}(2\pi(500t - F_ct))\]

And finally, factoring out t from each term.

  \[ x_{TX}(t) = \dfrac{1}{2}\text{cos}(2\pi t(500 + F_c)) + \dfrac{1}{2}\text{cos}(2\pi t(500 - F_c)) \]


\clearpage

\subsection*{Calculation 1 Results:}

As shown through using trigonometic identities, and basic algebraic techniques, \(x_{TX}(t)\) can be expressed as a sum of cosines as follows.

\begin{equation}
   x_{TX}(t) = \dfrac{1}{2}\text{cos}(2\pi t(500 + F_c)) + \dfrac{1}{2}\text{cos}(2\pi t(500 - F_c)) 
\end{equation}


\subsection*{Calculation 2: The Fourier Transform of \(x_{TX}(t)\)}

The Fourier Transform for a continuous signal is defined as

\begin{equation}   
    X(F) = \int_{-\infty}^{\infty} x(t) e^{-j2\pi Ft}dt
\end{equation}

Thankfully, the function \(x_{TX}(t)\) has a known Fourier Transform pair. The known pairing is

\begin{equation}
    x(t) = \text{cos}(2\pi F_0t) \xleftrightarrow{\quad \mathcal{F} \quad}X(F) = \dfrac{1}{2}[\delta(F - F_0) + \delta(F +F_0)]
\end{equation}
\newline
\newline
Since a property of the Fourier Transform is linearity, this means that we can split the function into two parts and apply the transform to each part. The first cosine term is
\begin{equation}
\dfrac{1}{2}\text{cos}(2\pi t(500 + F_c)) \xleftrightarrow{\quad \mathcal{F} \quad} \dfrac{1}{2} \left[\dfrac{1}{2}[\delta(F - 500+F_c) + \delta(F +500+F_c)]\right]
\end{equation}
\[= \dfrac{1}{4} \left[ \delta (F - 500 +F_c) + \delta (F + 500 + F_c)\right]\]
\[= \dfrac{1}{4} \delta (F - 500 + F_c) + \dfrac{1}{4} \delta (F + 500 + F_c)\]

Then the second cosine term,

\begin{equation}
\dfrac{1}{2}\text{cos}(2\pi t(500 - F_c)) \xleftrightarrow{\quad \mathcal{F} \quad} \dfrac{1}{2} \left[\dfrac{1}{2}[\delta(F - 500-F_c) + \delta(F +500-F_c)]\right]
\end{equation}
\[= \dfrac{1}{4} \left[ \delta (F - 500 - F_c) + \delta (F + 500 - F_c)\right]\]
\[= \dfrac{1}{4} \delta (F - 500 - F_c) + \dfrac{1}{4} \delta (F + 500 - F_c)\]

Then, referencing back to the linearity property, we are able to add these two transformed terms back together to have the final Fourier Transform, \(X_{TX}(F)\).

\[X_{TX}(F) =\dfrac{1}{4} \delta (F - 500 + F_c) + \dfrac{1}{4} \delta (F + 500 + F_c) + \dfrac{1}{4} \delta (F - 500 - F_c) + \dfrac{1}{4} \delta (F + 500 - F_c)\]

Finally, we can factor out the \(1/4\) term from each cosine, giving us our final answer.

\begin{equation}
    X_{TX}(F) =\dfrac{1}{4} \left[ \delta (F - 500 + F_c) + \delta (F + 500 + F_c) + \delta (F - 500 - F_c) + \delta (F + 500 - F_c) \right]
\end{equation}

\subsection*{Calculation 2 Results:}

Using the linearity property and a known pair, we were able to perform the Fourier Transform on a signal consisting of a sum of cosines. The resulting signal is a sum of \(\delta\) functions, with
a magnitude of \(1/4\).


   \[x_{TX}(t) \xleftrightarrow{\quad \mathcal{F} \quad} X_{TX}(F) =\dfrac{1}{4} \left[ \delta (F - 500 + F_c) + \delta (F + 500 + F_c) + \delta (F - 500 - F_c) + \delta (F + 500 - F_c) \right]\]

\subsection*{Calculation 3:}

It is important to verify that the pairs and properties of the Fourier Transform were applied properly. In order to do this, the Fourier Transforms of both \(x(t)\) and \(x_{c}(t)\)
will be performed. Beginning with \(x_t(t)\), we can note that the signal can be used with the same pair used in finding the Fourier Transform of \(X_{TX}(F)\) being,


    \[x(t) = \text{cos}(2\pi F_0t) \xleftrightarrow{\quad \mathcal{F} \quad}X(F) = \dfrac{1}{2}[\delta(F - F_0) + \delta(F +F_0)]\]

Substituting in our \(x(t)\) gives
\[x(t) = \text{cos}(2\pi \times 500t) \xleftrightarrow{\quad \mathcal{F} \quad}X(F) = \dfrac{1}{2}[\delta(F - 500) + \delta(F +500)]\]
\[x(t) = \text{cos}(2\pi \times 500t) \xleftrightarrow{\quad \mathcal{F} \quad}X(F) = \dfrac{1}{2}\delta(F - 500) + \dfrac{1}{2}\delta(F +500)\]
and doing the same for \(x_c(t)\)
\[x_c(t) = \text{cos}(2\pi F_ct) \xleftrightarrow{\quad \mathcal{F} \quad}X_c(F) = \dfrac{1}{2}[\delta(F - Fc) + \delta(F + Fc)]\]
\[x_c(t) = \text{cos}(2\pi F_ct) \xleftrightarrow{\quad \mathcal{F} \quad}X_c(F) = \dfrac{1}{2}\delta(F - Fc) + \dfrac{1}{2}\delta(F + Fc)\]
With these two Fourier Transforms, and the multiplication property,
\[x(t)y(t) \xleftrightarrow{\quad \mathcal{F} \quad} X(F) \ast Y(F) \]
we can define the Fourier Transform of \(x_{TX}(t)\) as
\[X_{TX}(F) = X(F) \ast X_C(F)\]

Substituing the values found above for \(X(F)\) and \(X_c(F)\), we have the new function
\[X_{TX}(F) = \dfrac{1}{2}\delta(F - 500) + \dfrac{1}{2}\delta(F +500) \ast \dfrac{1}{2}\delta(F - Fc) + \dfrac{1}{2}\delta(F + Fc) \]

To simplify the terms of this \(X_{TX}(F)\), the function must be colvolved. To do this, the shifting and linearity properties of convolution must be applied.
\[x(t) \ast \delta(t - t_0) = x(t - t_0)\]
\clearpage

For ease of viewing, each term in the convolution has been labeled as follows

\begin{equation}
    X_{TX}(F) = \overset{\circled{1}}{ {\frac{1}{2}\delta(F - 500)} } 
              + \overset{\circled{2}}{ {\frac{1}{2}\delta(F + 500)} } 
              \ast \overset{\circled{3}}{ {\frac{1}{2}\delta(F - F_c)} } 
              + \overset{\circled{4}}{ {\frac{1}{2}\delta(F + F_c)} }
\end{equation}

each term on the right hand side, \circled{3} and \circled{4}, must be applied to each term on the left hand side, \circled{1} and \circled{2}.

\[\overset{\circled{1}}{ {\frac{1}{2}\delta(F - 500)}} \ast \overset{\circled{3}}{ {\frac{1}{2}\delta(F - F_c)} }  = \dfrac{1}{4} \delta(F - 500 - F_C)\]
\[\overset{\circled{1}}{ {\frac{1}{2}\delta(F - 500)} } \ast \overset{\circled{4}}{ {\frac{1}{2}\delta(F + F_c)} } = \dfrac{1}{4} \delta(F - 500 + F_C)\]
\[\overset{\circled{2}}{ {\frac{1}{2}\delta(F + 500)} }  \ast\overset{\circled{3}}{ {\frac{1}{2}\delta(F - F_c)} } = \dfrac{1}{4} \delta(F + 500 - F_C)\]
\[\overset{\circled{2}}{ {\frac{1}{2}\delta(F + 500)} }  \ast \overset{\circled{4}}{ {\frac{1}{2}\delta(F + F_c)} } = \dfrac{1}{4} \delta(F + 500 - F_C)\]



Then, referencing back to the linearity property, we are able to add these terms back together to have the final Fourier Transform, \(X_{TX}(F)\).

\[X_{TX}(F) =\dfrac{1}{4} \delta (F - 500 + F_c) + \dfrac{1}{4} \delta (F + 500 + F_c) + \dfrac{1}{4} \delta (F - 500 - F_c) + \dfrac{1}{4} \delta (F + 500 - F_c)\]

And we can factor out the \(1/4\) term from each cosine, giving us our final answer.

\begin{equation}
    X_{TX}(F) =\dfrac{1}{4} \left[ \delta (F - 500 + F_c) + \delta (F + 500 + F_c) + \delta (F - 500 - F_c) + \delta (F + 500 - F_c) \right]
\end{equation}
%%%%%%%%%%%%%%%%%%%%%%%%%%%%%%%%%%%%%%%%%%%%%%%%%%%%
% DOCUMENT END
%%%%%%%%%%%%%%%%%%%%%%%%%%%%%%%%%%%%%%%%%%%%%%%%%%%%
\end{document}