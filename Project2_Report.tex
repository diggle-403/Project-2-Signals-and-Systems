%--------------------------------------------
%%%%%%%%%%%%%%%%%%%%%%%%%%%%%%%%%%%%%%%%%%%%%%%%%%%%
% THE PREAMBLE
%%%%%%%%%%%%%%%%%%%%%%%%%%%%%%%%%%%%%%%%%%%%%%%%%%%%
%--------------------------------------------

%%%%%%%%%%%%%%%%%%%%%%%%%%%%%%%%%%%%%%%%%%%%%%%%%%%%
% DOCUMENT CLASS (don't ever change this unless specified to do so).
%%%%%%%%%%%%%%%%%%%%%%%%%%%%%%%%%%%%%%%%%%%%%%%%%%%% 
\documentclass[11pt]{article}


%%%%%%%%%%%%%%%%%%%%%%%%%%%%%%%%%%%%%%%%%%%%%%%%%%%%
% USEFUL PACKAGES THAT YOU NEED. (Everything here needs to be here for the program to run.  Do not delete anything from the preamble, but you can add new things that you find on the internet.)
%%%%%%%%%%%%%%%%%%%%%%%%%%%%%%%%%%%%%%%%%%%%%%%%%%%%
\usepackage{fullpage}
\usepackage{amsfonts}
\usepackage{amsmath}
\usepackage{amsthm}
\usepackage{graphicx}
\usepackage{color}
\usepackage{amssymb}
\usepackage{empheq}
\usepackage{mathrsfs}
\usepackage{enumerate}
\usepackage{tikz}
\usepackage{pgflibraryarrows}
\usepackage{pgflibrarysnakes}
\usepackage{upgreek}
\usepackage{tipa}
\usepackage{gensymb}
\usepackage{multicol}
\usepackage{textcomp}
\usepackage{float}
\usepackage{ulem}
\usepackage{setspace}

\usepackage{amsmath}
\usepackage{graphicx}
\usepackage{wrapfig}
\usepackage{matlab-prettifier}
\usepackage{listings}
% Define a custom style for MATLAB code using matlab-prettifier
\lstdefinestyle{Matlab-editor}{
    style=Matlab-Pyglike, % Use a nice default style from the package
    basicstyle=\mlttfamily\small, % Use a small, typewriter font
    frame=single,                 % Add a frame around the code
    numbers=left,                 % Add line numbers on the left
    numberstyle=\tiny\color{gray} % Make line numbers small and gray
}
\usepackage{tikz}
\newcommand*\circled[1]{\tikz[baseline=(char.base)]{
    \node[shape=circle,draw,inner sep=1pt] (char) {#1};}}
% Tables: nicer rules and multi-row cells
\usepackage{booktabs}
\usepackage{multirow}
\usepackage{pifont}
\newcommand{\cmark}{\ding{51}} % check
\newcommand{\xmark}{\ding{55}} % cross
\setlength{\parindent}{0pt}

%%%%%%%%%%%%%%%%%%%%%%%%%%%%%%%%%%%%%%%%%%%%%%%%%%%%
% CREATES A SOLUTION ENVIRONMENT FOR HOMEWORK ASSIGNMENTS (You may not use this but leave it here for future assignments)
%%%%%%%%%%%%%%%%%%%%%%%%%%%%%%%%%%%%%%%%%%%%%%%%%%%%
\usepackage{versions}
\excludeversion{sol}
\includeversion{sol}
\newenvironment{solution}{
\sol\\{\sc{Solution:}}}{
$\hfill\blacksquare$\endsol}


%%%%%%%%%%%%%%%%%%%%%%%%%%%%%%%%%%%%%%%%%%%%%%%%%%%%
% CUSTOM DEFINITIONS FOR CREATING DERIVATIVE SYMBOLS (You will definitely use these.)
%%%%%%%%%%%%%%%%%%%%%%%%%%%%%%%%%%%%%%%%%%%%%%%%%%%%
\newcommand{\pd}[2]{\frac{\partial #1}{\partial #2}}
\newcommand{\pdd}[2]{\frac{\partial^2 #1}{\partial {#2}^2}}
\newcommand{\de}[2]{\frac{d #1}{d #2}}
\newcommand{\dde}[2]{\frac{d^2 #1}{d #2^2}}
\renewcommand{\figurename}{Plot}

%%%%%%%%%%%%%%%%%%%%%%%%%%%%%%%%%%%%%%%%%%%%%%%%%%%%
% CREATES THE TITLE PAGE (Note where to put the author, title, and date.)
%%%%%%%%%%%%%%%%%%%%%%%%%%%%%%%%%%%%%%%%%%%%%%%%%%%%
\title{\textsc{Dylan Huston \\ Isai Moreno}}
\author{\Large{ECE 3793 -- \textsc{Project 2}}}
\date{November 18, 2025}


%%%%%%%%%%%%%%%%%%%%%%%%%%%%%%%%%%%%%%%%%%%%%%%%%%%%
% CREATES A HEADER FOR EACH PAGE.  Please note where the left, center, and right titles are:
%%%%%%%%%%%%%%%%%%%%%%%%%%%%%%%%%%%%%%%%%%%%%%%%%%%%
\usepackage{fancyhdr}
\setlength{\headheight}{15.2pt}
\pagestyle{fancy}
\setlength\headsep{30pt}
\lhead{Dylan Huston | Isai Moreno}   					
\chead{\textsc{Project 2}}			    %  Title in the center.
\rhead{\today}							%  Date on the right header.

\begin{document}

\maketitle
\pagebreak

%--------------------------------------------
%%%%%%%%%%%%%%%%%%%%%%%%%%%%%%%%%%%%%%%%%%%%%%%%%%%%
% END OF THE PREAMBLE AND BEGINNING OF THE ACTUAL DOCUMENT
%%%%%%%%%%%%%%%%%%%%%%%%%%%%%%%%%%%%%%%%%%%%%%%%%%%%
%--------------------------------------------


%%%%%%%%%%%%%%%%%%%%%%%%%%%%%%%%%%%%%%%%%%%%%%%%%%%%
% SECTIONS, SUBSECTIONS, EQUATIONS, and OTHER THINGS:
%%%%%%%%%%%%%%%%%%%%%%%%%%%%%%%%%%%%%%%%%%%%%%%%%%%%

\section*{Introduction}

In this project, our team will be deriving and applying Fourier constants across a simple step function. During this process several MATLAB calculations will be performed, as well as deriving the Fourier constant integral to be used in recreating a signal. These signals and constants will then be plotted. The code that is used to derive the Fourier constants will then be used to derive the constants of different instruments playing the C3 note.  


\pagebreak

\section*{Part I(a): Derivation for the Transmitter}

Given a signal with a frequency of 500 Hz, \\ 
\begin{equation}
    x(t) = \text{cos}(2\pi \times 500t)
\end{equation}
and an arbitrary carrier frequency (\(F_c\)), to which the carrier signal is
\begin{equation}
    x_c(t) = \text{cos}(2\pi F_ct)
\end{equation}

\subsection*{Calculation 1: \(x_{TX}(t)\) as a sum of cosines}

\begin{equation}
x_{TX}(t) = x(t)x_c(t)
\end{equation}

Using the known signals, \(x(t)\) and \(x_c(t)\), we can substitute these functions into \(x_{TX}(t)\).

\[x_{TX}(t) = (\text{cos}(2\pi \times 500t))(\text{cos}(2\pi F_ct))\]

Using the trigonometric identity
\begin{equation} 
\text{cos}(u)\text{cos}(v) = \dfrac{1}{2}[\text{cos}(u+v)+\text{cos}(u-v)]
\end{equation}

Substituting in equation 1 for the cos(u) term, and equation 2 for the cos(v) term,

   \[\text{cos}(2\pi \times 500)\text{cos}(2\pi F_ct) = \dfrac{1}{2}[\text{cos}(2\pi \times 500t + 2\pi F_ct) + \text{cos}(2\pi \times 500t - 2\pi F_ct)]\]


Resulting in the equation

\begin{equation}   
x_{TX}(t) = \dfrac{1}{2}[\text{cos}(2\pi \times 500t + 2\pi F_ct) + \text{cos}(2\pi \times 500t - 2\pi F_ct)]
\end{equation}

Then distributing the \(1/2\) to each cosine

\[x_{TX}(t) = \dfrac{1}{2}\text{cos}(2\pi \times 500t + 2\pi F_ct) + \dfrac{1}{2}\text{cos}(2\pi \times 500t - 2\pi F_ct)\]

and factoring out \(2\pi\) from each cosine term

\[x_{TX}(t) = \dfrac{1}{2}\text{cos}(2\pi(500t + F_ct)) + \dfrac{1}{2}\text{cos}(2\pi(500t - F_ct))\]

And finally, factoring out t from each term.

  \[ x_{TX}(t) = \dfrac{1}{2}\text{cos}(2\pi t(500 + F_c)) + \dfrac{1}{2}\text{cos}(2\pi t(500 - F_c)) \]


\clearpage

\subsection*{Calculation 1 Results:}

As shown through using trigonometic identities, and basic algebraic techniques, \(x_{TX}(t)\) can be expressed as a sum of cosines as follows.

\begin{equation}
   x_{TX}(t) = \dfrac{1}{2}\text{cos}(2\pi t(500 + F_c)) + \dfrac{1}{2}\text{cos}(2\pi t(500 - F_c)) 
\end{equation}


\subsection*{Calculation 2: The Fourier Transform of \(x_{TX}(t)\)}

The Fourier Transform for a continuous signal is defined as

\begin{equation}   
    X(F) = \int_{-\infty}^{\infty} x(t) e^{-j2\pi Ft}dt
\end{equation}

Thankfully, the function \(x_{TX}(t)\) has a known Fourier Transform pair. The known pairing is

\begin{equation}
    x(t) = \text{cos}(2\pi F_0t) \xleftrightarrow{\quad \mathcal{F} \quad}X(F) = \dfrac{1}{2}[\delta(F - F_0) + \delta(F +F_0)]
\end{equation}
\newline
\newline
Since a property of the Fourier Transform is linearity, this means that we can split the function into two parts and apply the transform to each part. The first cosine term is
\begin{equation}
\dfrac{1}{2}\text{cos}(2\pi t(500 + F_c)) \xleftrightarrow{\quad \mathcal{F} \quad} \dfrac{1}{2} \left[\dfrac{1}{2}[\delta(F - 500+F_c) + \delta(F +500+F_c)]\right]
\end{equation}
\[= \dfrac{1}{4} \left[ \delta (F - 500 +F_c) + \delta (F + 500 + F_c)\right]\]
\[= \dfrac{1}{4} \delta (F - 500 + F_c) + \dfrac{1}{4} \delta (F + 500 + F_c)\]

Then the second cosine term,

\begin{equation}
\dfrac{1}{2}\text{cos}(2\pi t(500 - F_c)) \xleftrightarrow{\quad \mathcal{F} \quad} \dfrac{1}{2} \left[\dfrac{1}{2}[\delta(F - 500-F_c) + \delta(F +500-F_c)]\right]
\end{equation}
\[= \dfrac{1}{4} \left[ \delta (F - 500 - F_c) + \delta (F + 500 - F_c)\right]\]
\[= \dfrac{1}{4} \delta (F - 500 - F_c) + \dfrac{1}{4} \delta (F + 500 - F_c)\]

Then, referencing back to the linearity property, we are able to add these two transformed terms back together to have the final Fourier Transform, \(X_{TX}(F)\).

\[X_{TX}(F) =\dfrac{1}{4} \delta (F - 500 + F_c) + \dfrac{1}{4} \delta (F + 500 + F_c) + \dfrac{1}{4} \delta (F - 500 - F_c) + \dfrac{1}{4} \delta (F + 500 - F_c)\]

Finally, we can factor out the \(1/4\) term from each cosine, giving us our final answer.

\begin{equation}
    X_{TX}(F) =\dfrac{1}{4} \left[ \delta (F - 500 + F_c) + \delta (F + 500 + F_c) + \delta (F - 500 - F_c) + \delta (F + 500 - F_c) \right]
\end{equation}

\subsection*{Calculation 2 Results:}

Using the linearity property and a known pair, we were able to perform the Fourier Transform on a signal consisting of a sum of cosines. The resulting signal is a sum of \(\delta\) functions, with
a magnitude of \(1/4\).


   \[x_{TX}(t) \xleftrightarrow{\quad \mathcal{F} \quad} X_{TX}(F) =\dfrac{1}{4} \left[ \delta (F - 500 + F_c) + \delta (F + 500 + F_c) + \delta (F - 500 - F_c) + \delta (F + 500 - F_c) \right]\]

\subsection*{Calculation 3:}

It is important to verify that the pairs and properties of the Fourier Transform were applied properly. In order to do this, the Fourier Transforms of both \(x(t)\) and \(x_{c}(t)\)
will be performed. Beginning with \(x_t(t)\), we can note that the signal can be used with the same pair used in finding the Fourier Transform of \(X_{TX}(F)\) being,


    \[x(t) = \text{cos}(2\pi F_0t) \xleftrightarrow{\quad \mathcal{F} \quad}X(F) = \dfrac{1}{2}[\delta(F - F_0) + \delta(F +F_0)]\]

Substituting in our \(x(t)\) gives
\[x(t) = \text{cos}(2\pi \times 500t) \xleftrightarrow{\quad \mathcal{F} \quad}X(F) = \dfrac{1}{2}[\delta(F - 500) + \delta(F +500)]\]
\[x(t) = \text{cos}(2\pi \times 500t) \xleftrightarrow{\quad \mathcal{F} \quad}X(F) = \dfrac{1}{2}\delta(F - 500) + \dfrac{1}{2}\delta(F +500)\]
and doing the same for \(x_c(t)\)
\[x_c(t) = \text{cos}(2\pi F_ct) \xleftrightarrow{\quad \mathcal{F} \quad}X_c(F) = \dfrac{1}{2}[\delta(F - Fc) + \delta(F + Fc)]\]
\[x_c(t) = \text{cos}(2\pi F_ct) \xleftrightarrow{\quad \mathcal{F} \quad}X_c(F) = \dfrac{1}{2}\delta(F - Fc) + \dfrac{1}{2}\delta(F + Fc)\]

This results in our final calculations for \(X(F)\) and \(X_c(F)\) to be 
\[X(F) = \dfrac{1}{2}\delta(F - 500) + \dfrac{1}{2}\delta(F +500)\]
\[X_c(F) = \dfrac{1}{2}\delta(F - Fc) + \dfrac{1}{2}\delta(F + Fc)\]

\clearpage
\subsection*{Calculation 4:}
With these two Fourier Transforms, and the multiplication property,
\[x(t)y(t) \xleftrightarrow{\quad \mathcal{F} \quad} X(F) \ast Y(F) \]
we can define the Fourier Transform of \(x_{TX}(t)\) as
\[X_{TX}(F) = X(F) \ast X_C(F)\]

Substituing the values found above for \(X(F)\) and \(X_c(F)\), we have the new function
\[X_{TX}(F) = \dfrac{1}{2}\delta(F - 500) + \dfrac{1}{2}\delta(F +500) \ast \dfrac{1}{2}\delta(F - Fc) + \dfrac{1}{2}\delta(F + Fc) \]

To simplify the terms of this \(X_{TX}(F)\), the function must be convolved. To do this, the shifting and linearity properties of convolution must be applied.
\[x(t) \ast \delta(t - t_0) = x(t - t_0)\]

For ease of viewing, each term in the convolution has been labeled as follows

\begin{equation}
    X_{TX}(F) = \overset{\circled{1}}{ {\frac{1}{2}\delta(F - 500)} } 
              + \overset{\circled{2}}{ {\frac{1}{2}\delta(F + 500)} } 
              \ast \overset{\circled{3}}{ {\frac{1}{2}\delta(F - F_c)} } 
              + \overset{\circled{4}}{ {\frac{1}{2}\delta(F + F_c)} }
\end{equation}

each term on the right hand side, \circled{3} and \circled{4}, must be applied to each term on the left hand side, \circled{1} and \circled{2}.

\[\overset{\circled{1}}{ {\frac{1}{2}\delta(F - 500)}} \ast \overset{\circled{3}}{ {\frac{1}{2}\delta(F - F_c)} }  = \dfrac{1}{4} \delta(F - 500 - F_C)\]
\[\overset{\circled{1}}{ {\frac{1}{2}\delta(F - 500)} } \ast \overset{\circled{4}}{ {\frac{1}{2}\delta(F + F_c)} } = \dfrac{1}{4} \delta(F - 500 + F_C)\]
\[\overset{\circled{2}}{ {\frac{1}{2}\delta(F + 500)} }  \ast\overset{\circled{3}}{ {\frac{1}{2}\delta(F - F_c)} } = \dfrac{1}{4} \delta(F + 500 - F_C)\]
\[\overset{\circled{2}}{ {\frac{1}{2}\delta(F + 500)} }  \ast \overset{\circled{4}}{ {\frac{1}{2}\delta(F + F_c)} } = \dfrac{1}{4} \delta(F + 500 + F_C)\]



Then, referencing back to the linearity property, we are able to add these terms back together to have the final Fourier Transform, \(X_{TX}(F)\).

\[X_{TX}(F) =\dfrac{1}{4} \delta (F - 500 + F_c) + \dfrac{1}{4} \delta (F + 500 + F_c) + \dfrac{1}{4} \delta (F - 500 - F_c) + \dfrac{1}{4} \delta (F + 500 - F_c)\]

And we can factor out the \(1/4\) term from each cosine, giving us our final answer.

\begin{equation}
    X_{TX}(F) =\dfrac{1}{4} \left[ \delta (F - 500 + F_c) + \delta (F + 500 + F_c) + \delta (F - 500 - F_c) + \delta (F + 500 - F_c) \right]
\end{equation}

\clearpage

Comparing this to the \(X_{TX}(F)\) found in \textbf{Calculation 2}, we can see that both of these ways result in the same answer, verifying the linearity and convolution properties of the
Fourier Transform. 

\subsection*{Question 1: How is \(X_{TX}(F)\) related to \(X(F)\)?}

Let's begin by looking at a graph of \(X(F)\).
\begin{figure}[H]
        \centering
        \includegraphics[width=1\textwidth]{Question1_XF.png}  
    \end{figure}

Above we can see two stems, located at -500 and +500, with a height of 1/2. With the signal being centered around 0. Now let us look at a graph for \(X_{TX}(F)\), with an \(F_c\) of 0.
\begin{figure}[H]
        \centering
        \includegraphics[width=1\textwidth]{Q1_FC0.png}  
    \end{figure}

    We can see that this is the same graph, only scaled down due to the additional 1/2 from \(X_c(F)\). Looking at a graph where \(F_c = 5500\), we can see a change in the graph. 

    \begin{figure}[H]
        \centering
        \includegraphics[width=0.9\textwidth]{Q1_FC5500.png}  
    \end{figure}

    Looking at this graph, we can see that when the signal is shifted by $F_c$, it is bound by \(X(F)\). 
    \section*{WE AINT DONE COME BACK HERE}


    Now, should we be given an audio waveform \(x_a(t)\), given only the knowledge that it has a Fourier transform \(X_a(F)\) which only has non-zero values for \(\left|F\right| < 22\)kHz
    and it is a real signal, meaning that it is conjugate symmetric.\\

    We shall now redefine our transmit signal by
    \[x_{TX}(t) = x_a(t)x_c(t)\]

    \subsection*{Calculation 5:}
    Despite not knowing the signal \(x_a(t)\), we will still take the Fourier transform to find our new, transformed signal, \(X_{TX}(F)\). For now, we will define the Fourier transform of \(x_a(t)\) as \(X_a(F)\).
    \[x_{TX}(t) = x_a(t)x_c(t) \xleftrightarrow{\quad \mathcal{F} \quad} X_{TX}(F) = X_a(F)\ast X_c(F)\]
    Since we have previously defined \(X_c(F)\), we can easily substitute it in.
    \[X_{TX}(F) = X_a(F) \ast \dfrac{1}{2}\delta(F - Fc) + \dfrac{1}{2}\delta(F + Fc)\]
    Now, using the shifting property of convolution, \(x(t) \ast \delta(t-t_0) = x(t-t_0)\).

    \[X_{TX}(F) = \dfrac{1}{2}X_a(F-F_c) + \dfrac{1}{2}X_a(F+F_c)\]

    \subsection*{Question 2: How is \(X_{TX}(F)\) related to \(X_a(F)\)?}

    \section*{I AINT DONE WITH YOU BOY}

    \subsection*{Question 3: What is the bandwidth of \(X_a(F)\) if we only consider the positive side of the F-Axis?}
    With the hypothetical that the minimum frequency of \(X_a(F)\) is F = 0 Hz, and the maximum frequency is F = 22kHz, the bandwith is the difference between the largest and smallest frequency.
    The difference between the two frequencies, 0 Hz and 22k Hz, is 22k Hz. With that
    we can say that \textbf{the bandwith of \(X_a(F)\) is 22kHz}

    \subsection*{Question 4: What is the bandwidth of \(X_{TX}(F)\) if we only consider the positive side of the F-Axis?}

    \section*{NOT DONE YET}

     \subsection*{Question 5: How does the bandwidth of \(X_{TX}(F)\) compare with the bandwidth of \(X_a(F)\)}

    \section*{NOT DONE YET}

\clearpage
    \section*{Part I(b): Derivation for the Receiver}

    A received signal, \(x_{RX}(t)\) is defined as
    \[x_{RX}(t) = x_{TX}(t) + x_I(t)\]
where \(x_I(t)\) denotes interfering signals. We know nothing about \(x_I(t)\) other than it has a Fourier transform, \(X_I(F)\), and \(X_I(F)\) = 0 for all values of F where \(X_{TX}(F)\)
is non-zero. (out-of-band interferers.)

\subsection*{Calculation 6: Write an expression for the Fourier transform \(X_{RX}(F)\) in terms of \(X_{TX}(F)\) and \(X_I(F)\)}

\[x_{RX}(t) = x_{TX}(t) + x_I(t) \xleftrightarrow{\quad \mathcal{F}\quad} X_{RX}(F) = X_{TX}(F) + X_I(F)\]

\subsection*{Calculation 7: Next, replace the expression for \(X_{TX}(F)\) with its expression in terms of \(X_a(F)\)}
\(X_{TX}(F)\) is defined as
\[X_{TX}(F) = \dfrac{1}{2}X_a(F-F_c) + \dfrac{1}{2}X_a(F+F_c)\]
now substituing this into the equation for \(X_{RX}(F)\)
\[X_{RX}(F) = \dfrac{1}{2}X_a(F-F_c) + \dfrac{1}{2}X_a(F+F_c) + X_I(F)\]

Now, the signal \(x_{RX}(t)\) is down-converted from the carrier frequency (\(F_c\)) to the audio frequency by multiplying it by \(x_c(t)\). Let \(x_d(t)\) be the down-converted signal, given by
\[x_d(t) = x_{RX}(t)x_c(t)\]

\subsection*{Calculation 8: The Fourier Transform of \(x_d(t)\)}

Using the multiplication property of the Fourier transform, 

 \[x(t)y(t) \xleftrightarrow{\quad \mathcal{F} \quad} X(F)\ast Y(F)\]
 We can define \(X_d(F)\) as

 \[X_d(F) = X_{RX}(F) \ast X_c(F)\]

 \(X_{RX}(F)\) is defined as

\[X_{RX}(F) = \dfrac{1}{2}X_a(F-F_c) + \dfrac{1}{2}X_a(F+F_c) + X_I(F)\]

Allowing us to substitute into the equation for \(X_d(F)\)

\[X_d(F) = \left[\dfrac{1}{2}X_a(F-F_c) + \dfrac{1}{2}X_a(F+F_c) + X_I(F) \right]\ast X_c(F)\]

and \(X_c(F)\) is defined as
\[X_c(F) = \dfrac{1}{2}\delta(F - Fc) + \dfrac{1}{2}\delta(F + Fc)\]

Allowing another substitution.

\[X_d(F) = \left[\dfrac{1}{2}X_a(F-F_c) + \dfrac{1}{2}X_a(F+F_c) + X_I(F) \right]\ast \left[\dfrac{1}{2}\delta(F - Fc) + \dfrac{1}{2}\delta(F + Fc)\right]\]

Using the shifting property of convolution,

\[X_d(F) = \dfrac{1}{4}X_a(F - F_c - F_c) + \dfrac{1}{4}X_a(F + F_c - F_c) + \dfrac{1}{2}X_I(F - F_c) + \dfrac{1}{4}X_a(F - F_c + F_c) + \dfrac{1}{4}X_a(F + F_c + F_c) + \dfrac{1}{2}X_I(F+F_c)\]

Simplifying this gives

\[X_d(F) = \dfrac{1}{4}X_a(F-2F_c) + \dfrac{1}{4}X_a(F) + \dfrac{1}{2}X_I(F-F_c) + \dfrac{1}{4}X_a(F) + \dfrac{1}{4}X_a(F+2F_c) + \dfrac{1}{2}X_I(F+F_c)\]

Further simplifying and rearranging terms,

\[X_d(F) = \dfrac{1}{4}X_a(F-2F_c) + \dfrac{1}{4}X_a(F+2F_c) + \dfrac{1}{2}X_a(F) + \dfrac{1}{2}X_I(F+F_c) + \dfrac{1}{2}X_I(F-F_c)\]

Our final answer, and expression for \(X_d(F)\).

\subsection*{Calculation 8: The Inverse Fourier Transform of \(X_d(F)\)}

Using the Fourier pairing

\[x(t) = \text{cos}(2\pi F_0 t) \xleftrightarrow{\quad \mathcal{F} \quad} X(F) = \dfrac{1}{2}\left[ \delta (F-F_0) + \delta (F+F_0)\right]\]

we are able to simplify the \(X_a(F \pm 2F_c)\) and \(X_I(F \pm F_c)\) terms. Firstly, the \(X_a(F)\) terms.

\[\dfrac{1}{4}X_a(F - 2F_c) + \dfrac{1}{4}X_a(F + 2F_c)\]
\[\dfrac{1}{4} \Bigl[X_a(F - 2F_c) + X_a(F + 2F_c)  \Bigr] \]
\[\dfrac{1}{2}\biggl[ \dfrac{1}{2} \Bigl[X_a(F - 2F_c) + X_a(F + 2F_c)  \Bigr] \biggr]\]
\[\dfrac{1}{2}\biggl[ \dfrac{1}{2} \Bigl[X_a(F - 2F_c) + X_a(F + 2F_c)  \Bigr] \biggr]\xrightarrow{\quad \mathcal{F}^{-1} \quad} \dfrac{1}{2}x_a(t)\text{cos}(2\pi(2F_c)t)\]
\[X_a(F \pm 2F_c) \xrightarrow{\mathcal{F}^{-1}}\dfrac{1}{2}x_a(t)\text{cos}(4\pi F_ct)\]


Now, the \(X_I(F \pm F_c)\) terms.
\[\dfrac{1}{2}X_I(F + F_c) + \dfrac{1}{2}X_I(F - F_c)\]
\[\dfrac{1}{2} \Bigl[ X_I(F + F_c) + X_I(F - F_c) \Bigr]\]
\[\dfrac{1}{2} \Bigl[ X_I(F + F_c) + X_I(F - F_c) \Bigr] \xrightarrow{\quad \mathcal{F}^{-1} \quad} x_I(t)\text{cos}(2\pi F_ct)\]
\[X_I(F \pm F_0) \xrightarrow{\mathcal{F}^{-1}} x_I(t)\text{cos}(2\pi F_ct)\]

The final transform which has no shifts, is simple enough

\[\dfrac{1}{2}X_a(F) \xrightarrow{\mathcal{F}^{-1}} \dfrac{1}{2}x_a(t)\]

Now, with the pieces that we have found, we can reconstruct the signal using the linearity property of the Fourier transform.

\[x_d(t) = \dfrac{1}{2}x_a(t) + \dfrac{1}{2}x_a(t)\text{cos}(4\pi F_c t) + x_I(t)\text{cos}(2\pi F_c t)\]

\subsection*{Calculation 10: The Fourier transform for the ideal filter \(H(F)\)}
To fully recieve our signal, we must now apply a filter to remove the unwanted content, so we are only left with \(x_a(t)\). Knowing that our transformed signal, \(X_a(F)\) does not
have content above \(F = 22\)kHz. The ideal low pass filter is a rect function, centered at \(F = 0\)Hz, with a width of \(2F_{LPF}\) (Low-Pass Filter), where \(F_{LPF}\) is the cutoff
frequency of the filter. In our case, \(F_{LPF} = 22\)kHz

Knowing that we are centered at F = 0, and have the width of 2\(F_{LPF}\), with the known value of \(F_{LPF}\) = 22kHz we can derive the piecewise function for the shape we want.

\begin{equation*}
    H(F) =
\begin{cases}
    1 & \text{if } |F|\le 22\text{ kHz}\\
    0 & \text{if } |F|> 22\text{ kHz}\\
\end{cases}
\end{equation*}


This works because if F is in the range  \(-22\text{ kHz} \le F \le 22\text{ Hz} \), we will allow the signal through, but if we are outside the range, \(F > 22\text{ kHz}\) or \(F < -22 \text{ kHz}\)
nothing will pass through.

Using the piecewise function we created for H(F), we can define H(F) formally as

\[H(F) = \text{rect}\Bigl(\dfrac{F}{44\text{ kHz}}\Bigr)\]

\subsection*{Calculation 11: Determine the Impulse Response h(t)}

The rect function has known Fourier pairing,
\[X(F) = \text{rect}\Bigl(\dfrac{F}{W}\Bigr) \xleftrightarrow{\quad \mathcal{F} \quad } x(t) = W\text{sinc}(Wt)\]

Using our defined H(F)

\[H(F) =\text{rect}\Bigl(\dfrac{F}{44\text{ kHz}}\Bigr) \xleftrightarrow{\quad \mathcal{F} \quad } h(t) = 44\text{ kHz }\text{sinc}(44\text{ kHz }t)\]

Otherwise,

\[h(t) = 44000\text{ sinc}(44000t)\]
\clearpage
\subsection*{Calculation 12: The Fourier Transform \(X_{out}(F)\)}

\(x_{out}(t)\) is defined as

\[x_{out}(t) = x_d(t) \ast h(t)\]

Using the Fourier transform property of convolution,

\[x(t) \ast y(t) \xleftrightarrow{\mathcal{F}^{-1}} X(F)Y(F)\]

we can substitute in our known values of \(X_d(F)\) and \(H(F)\).



%%%%%%%%%%%%%%%%%%%%%%%%%%%%%%%%%%%%%%%%%%%%%%%%%%%%
% DOCUMENT END
%%%%%%%%%%%%%%%%%%%%%%%%%%%%%%%%%%%%%%%%%%%%%%%%%%%%
\end{document}